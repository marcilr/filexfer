%% -*- Mode: LaTeX -*-
%%
%% filexfer.tex
%% Created Mon Jun 27 09:39:25 AKDT 2016
%% by Raymond E. Marcil <rmarcil@gci.com>
%% 
%% FileXfer - File Transfer Jobs
%%

  %%
%%%%%% Preamble.
  %%

%% Specify DVIPS driver used by things like hyperref
\documentclass[12pt,letterpaper,dvips]{article}

%% rcs is the package to display cvs revision info.
%%\usepackage{rcs}
\usepackage{fullpage}
\usepackage{fancyvrb} 
\usepackage{graphicx}
\usepackage{figsize}
\usepackage{calc}

%%
%% enumitem – Control layout of itemize, enumerate, description
%% https://www.ctan.org/pkg/enumitem
%%
%% Allows for use of \bgein{itemize}[leftmargin=0pt] 
%% to lists with 0 left margin.
%%
%% Itemize left margin
%% http://tex.stackexchange.com/questions/170525/itemize-left-margin
%% 
\usepackage{enumitem}%     http://ctan.org/pkg/enumitem


%% caption package for use in justifying table or figure captions
\usepackage{caption}

\usepackage{xspace}
\usepackage{booktabs}
\usepackage[first,bottomafter]{draftcopy}
\usepackage[numbib]{tocbibind}

\usepackage{amssymb}              %% AMS Symbols, used for \checkmark
\usepackage{multicol}

%%
%% Extract SVN metadata for use elsewhere.
%% This information has:
%% o the filename
%% o the revision number
%% o the date and time of the last Subversion co command
%% o name of the user who has done the action
%%
%% FIXME: Need to update this for git.
%%
%%\usepackage{svninfo}
%%\svnInfo $Id: filexfer.tex 52 2016-06-27 09:43:54Z marcilr $


%%
%% Including Git Revision Identifiers in LaTeX
%% by Thore Husfeldt
%% https://thorehusfeldt.net/2011/05/13/including-git-revision-identifiers-in-latex/
%% Configure git user name
%%   For individual repo:
%%     $ git config user.name "Bob Jones"
%%
%%   For all repos:
%%     $ git config --global user.name "Bob Jones"
%%
%% Verify user name for repo:
%%   $ git config user.name
%%   Bob Jones
%%
%% Verify user name all repos:
%%   $ git config --global user.name
%%   Bob Jones
%%
%% Changing your username in Git only affects commits that
%% you make after your change.
%%
%% To rewrite your old commits, you can use git filter-branch[1]
%% to change the repository history to use your new username.
%% [1] https://help.github.com/articles/changing-author-info
%%
%%% This file is generated by Makefile.
%%% Do not edit this file!
%%%
\gdef\GITAuthorEmail{marcilr@gmail.com}
\gdef\GITRevision{0.0.1}
	\gdef\GITAbrHash{c35cce8}	\gdef\GITAuthorDate{June 28, 2016}	\gdef\GITAuthorName{Raymond E. Marcil}

%%
%% Hyperref package for embedding URLs for clickable links in PDFs, 
%% also specify PDF attributes here.
%%
%% The pdfborder={0 0 0} is what ellimated the blue box around the url
%% displayed by \href{}{}.
%%
%% The command pdfborder={0 0 1} would display a box with thickness of 1 pt.
%%
%% Hypertext marks in LATEX: a manual for hyperref
%% by Sebastian Rahtz and Heiko Oberdiek - November 2012
%% http://ctan.org/pkg/hyperref 
%% http://mirror.hmc.edu/ctan/macros/latex/contrib/hyperref/doc/manual.html
%%
\usepackage[
colorlinks,
linkcolor=blue,
%%colorlinks=false,
hyperindex=false,
urlcolor=blue,
pdfborder={0 0 0},
pdfauthor={Raymond E. Marcil},
pdftitle={FileXfer File Transfer Jobs},
pdfcreator={ps2pdf},
pdfsubject={FileXfer, file transfer jobs},
pdfkeywords={FileXfer, file transfer jobs}
]{hyperref}


%%
%% Extract RCS metadata for use elsewhere.
%% Jason figured this out, very cool.
%%
%%\RCS $Revision: 1.53 $
%%\RCS $Date: 2006/06/26 21:04:55 $


  %%
%%%%%% Customization.
  %%

% On letter paper with 10pt font the Verbatim environment has 65 columns.
% With 12pt font the environment has 62 columns.  Exceeding this will exceed
% the frame and will look ugly.  YHBW.  HAND.
\RecustomVerbatimEnvironment{Verbatim}{Verbatim}{frame=single}

\renewenvironment{description}
                 {\list{}{\labelwidth 0pt \iteminden-\leftmargin
                          \let\labelsep\hsize
                          \let\makelabel\descriptionlabel}}
                 {\endlist}
\renewcommand*\descriptionlabel[1]{\hspace\labelsep\sffamily\bfseries #1}


  %%
%%%%%% Commands.
  %%

\newcommand{\FIXME}[1]{\textsf{[FIXME: #1]}}
\newcommand{\cmd}[1]{\texttt{#1}}


%% Squeeze space above/below captions
\setlength{\abovecaptionskip}{4pt}   % 0.5cm as an example
\setlength{\belowcaptionskip}{4pt}   % 0.5cm as an example


%% Tex really adds a lot of whitespace to itemized 
%% lists so define a new command itemize* with a 
%% lot less whitespace.  Found this in the British
%% Tex faq.
\newenvironment{itemize*}%
  {\begin{itemize}%
    \setlength{\itemsep}{0pt}%
    \setlength{\parsep}{0pt}}%
  {\end{itemize}}

  
%%
%% Tex really adds a lot of whitespace to itemized 
%% lists so define a new command itemize* with a 
%% lot less whitespace.  Found this in the British
%% Tex faq.
%%
%% Tue Jun 23 13:22:04 AKDT 2015
%% =============================
%% Added [leftmargin=0.0mm] to set the left margin=0
%% This requires use of the enumitem package:
%%   \usepackage{enumitem}%     http://ctan.org/pkg/enumitem
%%
%% Itemize left margin
%% http://tex.stackexchange.com/questions/170525/itemize-left-margin
%%
\newenvironment{itemizenoleft*}%
  {\begin{itemize}[leftmargin=15.0pt]%
    \setlength{\itemsep}{0pt}%
    \setlength{\parsep}{0pt}}%
  {\end{itemize}}
  

%%
%% Tex really adds a lot of whitespace to itemized 
%% lists so define a new command enumerate* with a 
%% lot less whitespace.  Created using itemize*
%% pattern.  
%%
  \newenvironment{enumerate*}%
  {\begin{enumerate}%
    \setlength{\itemsep}{0pt}%
    \setlength{\parsep}{0pt}}%
  {\end{enumerate}}


%%
%% Tex really adds a lot of whitespace to itemized 
%% lists so define a new command enumerate* with a 
%% lot less whitespace.  Created using itemize*
%% pattern.  
%%
%% Tue Jun 23 13:22:04 AKDT 2015
%% =============================
%% Added [leftmargin=0.0mm] to set the left margin=0
%% This requires use of the enumitem package:
%%   \usepackage{enumitem}%     http://ctan.org/pkg/enumitem
%%
%% Itemize left margin
%% http://tex.stackexchange.com/questions/170525/itemize-left-margin
%%
\newenvironment{enumeratenoleft*}%
  {\begin{enumerate}[leftmargin=0.0mm]%
    \setlength{\itemsep}{0pt}%
    \setlength{\parsep}{0pt}}%
  {\end{enumerate}}


%% Squeeze space
\renewcommand\floatpagefraction{.9}
\renewcommand\topfraction{.9}
\renewcommand\bottomfraction{.9}
\renewcommand\textfraction{.1}   
\setcounter{totalnumber}{50}
\setcounter{topnumber}{50}
\setcounter{bottomnumber}{50}


  %%
%%%%%% Document.
  %%

\title{FileXfer\\ File Transfer Jobs}

\author{\GITAuthorName\\
        \texttt{$<$\GITAuthorEmail$>$}
%%        \texttt{$<$rmarcil@gci.com$>$}
}

% Display subversion revision and date under author on 1st page.
%%\date{Revision \svnInfoRevision
%%      \hspace{2pt}
%%      (\svnInfoLongDate)}

%%
%% Including Git Revision Identifiers in LaTeX
%% Thore Husfelt
%% https://thorehusfeldt.net/2011/05/13/including-git-revision-identifiers-in-latex/
%%
\date{Revision \GITRevision\hspace{7.5pt}(\GITAuthorDate)\\
               \texttt{\GITAbrHash}}

% Set date to RCS revision date
%%\date{Revision \RCSRevision
%%      \hspace{2pt}
%%      (\RCSDate)}


%%
%% Display SVN (subversion) version data at top right of 1st page,
%% This may be preferable to underneath the author.
%%
%%\rhead{Revision \svnInfoRevision\\
%%\svnInfoLongDate}


\begin{document}

\maketitle

\begin{abstract}
  \noindent The FileXfer application is a system for automated file transfer
  jobs for copying files.  ``There are 3 applications that make up the usage
  collection framework: \texttt{filexfer}, which does the actual file
  transfers; \texttt{filexfer-jobmonitor}, which is configured
  to monitor various aspects of jobs and create NMS alarms when necessary; and
  \texttt{filexfer-dataloader}, which bulk-loads file data into database
  tables. There are also housekeeping scripts called
  \texttt{filexfer-filearchive}, which keeps files in the data directory
  pruned and compressed, and \texttt{filexfer-fileunarchive}, which allows
  files to be pulled out of the archive so filexfer jobs can work with them
  again.''\footnote{\href{http://oss-wiki.operations.gci.com/dev/index.php/Usage_Collection\_Framework\_(filexfer)}{Usage Collection Framework (filexfer)}}
\end{abstract}

\vspace{2.0in}

%% Draw DNR logo and address at bottom of page
%%\begin{figure}[h]
%%        \hspace{0.32in}
%%        \SetFigLayout{1}{1}
%%        \begin{minipage}[b]{0.16\figwidth}
%%                \includegraphics[width=\textwidth]{dnr_bwlogo.eps}
%%        \end{minipage}
%%        \hspace{5pt}
%%        \begin{minipage}[b]{\figwidth}
%%                \bf{Alaska Department of Natural Resources}\\
%%                \small{\sf{Division of Support Services\\
%%                Land Records Information Section\\
%%                550 W. 7th Ave. Suite 706\\
%%                Anchorage, Alaska 99501}}
%%        \end{minipage}
%%\end{figure}

\newpage
\tableofcontents

\newpage
\listoffigures
\listoftables


%% =============== List of Abbreviations ===============
%% =============== List of Abbreviations ===============
\newpage
\setcounter{secnumdepth}{0}
\section{List of Definitions and Abbreviations}
\begin{itemize*}
  \item{\begin{bf}MOA\end{bf}} - Municipality of Anchorage

\end{itemize*}


%% ====================== Introduction ===========================
%% ====================== Introduction ===========================
\newpage
\section{Introduction}
The FileXfer system...


%% ===================== Job Scheduling ==========================
%% ===================== Job Scheduling ==========================
\section{Job Scheduling}
Jobs are scheduled using a web interface at
\texttt{nms.operations.gci.com/relevance}.  Navigate to the
``FileXfer'' application and click the ``File Transfer Jobs'' link.
Job execution happens on \texttt{prod-prov4-cdr1.operations.gci.com}.
A \texttt{cron} job executes every minute from
\texttt{/etc/cron.d/filexfer} to kick off the various \texttt{filexfer}
scripts.\footnote{\href{http://oss-wiki.operations.gci.com/dev/index.php/Usage_Collection\_Framework\_(filexfer)}{Usage
    Collection Framework (filexfer)}}


%% ================== Application Logging ======================
%% ================== Application Logging ======================
\section{Application Logging}
The filexfer applications log to the \texttt{/var/log/filexfer}
directory on \texttt{prod-prov4-cdr1.}\textbackslash\\
\texttt{operations.gci.com}. The parent \texttt{filexfer} jobs
log to \texttt{filexfer-get.log} and
\texttt{filexfer-}\textbackslash\\
\texttt{put.log}.
The \texttt{jobmonitor} and \texttt{dataloader} applications
log to \texttt{jobmonitor.log} and\\
\texttt{dataloader.log}, The \texttt{filexfer} applications
log to the \texttt{/var/log/filexfer} directory on
\texttt{prod-prov4-cdr1.operations.gci.com}. The parent
\texttt{filexfer} jobs
log to \texttt{filexfer-get.}\textbackslash\\
\texttt{log} and \texttt{filexfer-put.log}. The
\texttt{jobmonitor} and \texttt{dataloader} applications log to\\
\texttt{jobmonitor.log} and \texttt{dataloader.log}, respectively. Each file
transfer job is executed as a child process and gets its own log file. The
format is\\
\texttt{filexfer-}$\{$\texttt{neName}$\}$\texttt{-}$\{$\texttt{idJob}$\}$\texttt{-}$\{$\texttt{get,put}$\}$\texttt{.log}.\\
\\
By default, the jobs log at the warn level.  Adjust the level to info to get a
high-level view of the application's state.  Adjust log verbosity by modifying
the appropriate config file in \texttt{/etc/filexfer}.  The changes will take
effect after the next program execution.\\
\\
Errors are also logged to a database table which can be browsed in the
filexfer web interface under the '\texttt{Logs \& Errors}' view.  This view
includes messages logged at \texttt{warn}, \texttt{error}, and \texttt{fatal}
severity.\footnote{\href{http://oss-wiki.operations.gci.com/dev/index.php/Usage_Collection\_Framework\_(filexfer)}{Usage Collection Framework (filexfer)}}


%% ================= File Transfer Logging =====================
%% ================= File Transfer Logging =====================
\section{File Transfer Logging}
Every file transfer is recorded in a database table. There are
two reasons for this table: first, it tells \texttt{filexfer}
hich files have already been transferred, and second, it
provides an audit trail for SOX compliance. The table is
\texttt{filexfer.logs} on
\texttt{sadc-cdr-mysql1.operations.gci.com}. Use the
\texttt{filexfer.joblogs} view to easily find logs by job name
or network element ID.\\
\\
File transfer logs may also be viewed in the
'\texttt{Logs \& Errors}' page of the web interface.\footnote{\href{http://oss-wiki.operations.gci.com/dev/index.php/Usage_Collection\_Framework\_(filexfer)}{Usage Collection Framework (filexfer)}}


%% ======================= Dataloader ==========================
%% ======================= Dataloader ==========================
\section{Dataloader}
Dataloader jobs are configured using the web interface at
\texttt{nms.operations.gci.com/relevance}. Navigate to the
``\texttt{FileXfer}'' application and click the
``\texttt{Data Load Jobs}'' link. These jobs are executed every
minute as long as there are files in the load queue.\footnote{\href{http://oss-wiki.operations.gci.com/dev/index.php/Usage\_Collection\_Framework\_(filexfer)}{Usage Collection Framework (filexfer)}}



%% ======================== Examples =============================
%% ======================== Examples =============================
\newpage
\section{Examples}
Series of useful \LaTeX\ markup. Need to break out to 
separate examples.tex file.

\subsection{Escaping $<$ and $>$ Symbols}
To get \$$<$\$ or \$$>$\$ just wrap the symbols in \$ for math mode.

\subsection{Enumerate}
\begin{enumerate}
  \item{DNR} - Alaska State Department of Natural Resources
    \begin{itemize*}
      \item{HI} - Historical Index, not maintained since 1982
      \item{LE} - Land Estate, maintained by SGU
      \item{ME} - Mineral Estate, maintaind by SGU
    \end{itemize*}

  \item{Alaska State Surveys}
    \begin{itemize*}
      \item{ASBLT} - As-Built Survey
      \item{ASCS} - Cadastral Survey
    \end{itemize*}
\end{enumerate}


%% ======================= Comments =======================
%% ======================= Comments =======================
\subsection{Comments}
\begin{center}
\framebox{
\begin{minipage}[t]{0.9\textwidth}
\cmd{COMMENTS} Comment --- \emph{Sean Weems, Spring 2003}\\
We should get the \cmd{COMMENTS} column searchable via the 
landrecords application before we do much anything else -- shouldn't
be too hard.
\end{minipage}
}
\end{center}

\begin{center}
\framebox{
\begin{minipage}[t]{0.9\textwidth}
\emph{Errata: Plats spanning multiple sections}\\
A few anomalies can be observed in the \cmd{AKPLATS}
table. Specifically plats exist that span multiple sections. 
Since the table only has a single column, \cmd{SCODE}, 
that accepts a single section code, SGU (Status
Graphics Unit) has handled this problem by entering multiple 
rows in the table, each with a different section that point to
the same plat or file. Multiple section plats are indicated by  setting 
the \cmd{TCODE} column to the value \cmd{37}, and making an 
appropriate notation like \emph{Section 24-25-26-27} in the 
\cmd{REMARKS} column.\\
\FIXME{Perhaps the \cmd{SCODE} column should accept an array of sections?}

\end{minipage}
}
\end{center}


%% ====================== Footnotes ======================
%% ====================== Footnotes ======================
\clearpage
\newpage
\subsection{Footnotes}
See my
footnote\footnote{\href{http://www.google.com/search?q=latex+footnotes}{Search}
google for footnotes.}
generated with:

\begin{verbatim}
  \footnote{\href{http://www.google.com/search?q=latex+footnotes}
  {Search google for footnotes.}}
\end{verbatim}

\noindent GoogleGuide --- Linking to Search
Results.\footnote{GoogleGuide --- \href{http://www.googleguide.com/linking.html}{Linking to Search Results.}}


%% ====================== Hyperlinks ======================
%% ====================== Hyperlinks ======================
\subsection{Hyperlinks}
Use $\backslash$href\{\} to generate hyperlinks:

\begin{verbatim}
  \href{http://www.google.com}{Google}}
\end{verbatim}

\noindent Yields: \href{http://www.google.com}{Google}


%% ================ Table Examples ================
%% ================ Table Examples ================
\subsection{Table Examples}
\begin{table}[htb]
\begin{tabular}{|p{.25\textwidth}|p{.20\textwidth}|p{.47\textwidth}|}\hline 
Column Name&Type&Description\\ 
\hline
EQS&VARCHAR2(1)&!NULL map shows village selections\\
ITM\_COL&VARCHAR2(1)&USGS ITM column: 1-6\\
ITM\_ROW&VARCHAR2(1)&USGS ITM row: A-E\\
QMQ\_ABBR\_DNR&VARCHAR2(3)&Three character DNR abbreviation for the QMQ\\
RASTER\_FILENAME&VARCHAR2(50)&Physical path to file\\
RASTER\_PATHNAME&VARCHAR2(50)&URL path to PDF of map\\
SCODE&VARCHAR2(2)&Supplement map code: 1,2,3,...\\
COMMENTS&VARCHAR2(256)&Plat comments\\
\hline
\end{tabular}
\caption {\cmd{EASEMENTS\_17B} Table}
\label{table:easements_17b}
\end{table}

%% A simple table example
%% The htb attribute attempts to inline the table or figure
%% where you put it in the document
\begin{table}[htb]
\begin{center}
\begin{tabular*}{\textwidth}{@{}p{.25\textwidth}@{}p{.75\textwidth}}
\hline
\hline\\[-2.5ex]
XML element&Descripton\\
\hline
\hline\\[-1.5ex]   %% Trick to add whitespace after horizontal line
FNUM&US Survey file number\\ 
MERIDIAN&BLM meridian code\\
&\hspace{10pt}12 = Copper River\\
&\hspace{10pt}13 = Fairbanks\\
&\hspace{10pt}28 = Seward\\
&\hspace{10pt}44 = Kateel\\
&\hspace{10pt}45 = Umiat\\
TOWNSHIP&Five character Township code\\
RANGE&Five character Range code\\
PAGE&Survey page number 1,2,3,...\\
FILENAME&Relative path to file in direcory\\[1.5ex]
\hline
\end{tabular*}
\caption {USS XML index elements}
\label{table:uss-index}
\end{center}
\end{table}


%%
%% NOTE: tabular* takes table width argument
%%       Use of 'p' for left aligned column.
%%       Use of 'cp' to center column.
%%       The @{} works to remove an unwanted space.
%%
%% FIXME: Set width of centered columns?
%%
%% Links
%% =====
%% Getting to Grips with Latex - Tables
%% by Andrew Roberts
%% Source for @{\extracolsep{\fill} syntax.
%% http://www.andy-roberts.net/misc/latex/latextutorial4.html
%%
\begin{table}[htb]
\begin{center}
\begin{tabular*}{0.90\textwidth}{@{\extracolsep{\fill}}@{}l@{}c@{}c@{}c}
  \hline
  \hline\\[-2.5ex]
  col 1 & col 2 & col 3 & col 4 \\
  \hline
  \hline\\[-1.5ex]
  item 1 & item 2 & item 3 & item 4 \\
  \hline
  item 1  & item 2  & item 3  & item 4  \\
  \hline
\end{tabular*}
\caption {Demo}
\label{table:demo-index}
\end{center}
\end{table}


%% ================== Installed Daemons ===================
%% ================== Installed Daemons ===================
%% The htb attribute attempts to inline the table or figure
%% where you put it in the document
%%
%% NOTE: tabular* takes table width argument
%%       Use of 'p' for left aligned column.
%%       Use of 'cp' to center column.
%%       The @{} works to remove an unwanted space.
%%
%% FIXME: The ELM adn Elluminate Server columns are not centering.
%%        Center with fixed width in LaTeX?
%%
%% Links
%% =====
%% Getting to Grips with Latex - Tables
%% by Andrew Roberts
%% http://www.andy-roberts.net/misc/latex/latextutorial4.html
%%
%%
%% Configure caption on left using caption package.
%% The key to this working is placing the \captionsetup
%% just before the table or figure for which to manipulate
%% the package.
%%
%% Other potential potential options include:
%%   labelsep=newline
%%   labelfont=bf
%%
%% Links
%% =====
%% Raggedright and caption package
%% http://tex.stackexchange.com/questions/120411/raggedright-and-caption-package
%%
\captionsetup{%
    justification=raggedright,
    singlelinecheck=false
}

\begin{table}[htb]
%%\begin{center}
\begin{tabular*}{.80\textwidth}{@{\extracolsep{\fill}}@{}p{.25\textwidth}@{}c@{}c@{}c@{}c}
\hline
\hline\\[-2.5ex]
Virtual Machine&Apache&ELM&LM&Elluminate Server\\
\hline
\hline\\[-1.5ex]   %% Trick to add whitespace after horizontal line
\cmd{dcs-elive-prod01}&\,&x&x&x\\
\cmd{uaa-elive-dev01}&x&x&x&\\
\cmd{uaa-elive-server01}&\,&\,&\,&x\\[1.5ex]
\cmd{uaa-elive-prod01}&\,&x&x&x\\
\cmd{uaf-elive-prod01}&\,&x&x&x\\
\cmd{uas-elive-prod01}&\,&x&x&x\\
\hline
\end{tabular*}
\caption {Daemons}
\label{table:daemons-index}
%%\end{center}
\end{table}



\begin{table}[htb]
\begin{tabular}{|p{.16\textwidth}|p{.17\textwidth}|p{.62\textwidth}|}\hline 
Column Name&Type&Description\\ 
\hline
MTR&VARCHAR2(9)&Meridian, Township, Range, example: \emph{C026S054E}\\
QMQ&VARCHAR2(3)&Quarter Million Quadrangle code,\\
&&\hspace{10pt}example: \emph{DIL} (Dillingham quadrangle)\\
\hline
\end{tabular}
\caption {\cmd{XREF\_MTR\_QMQ} Table}
\label{table:xref_mtr_qmq}
\end{table}


%% ====================== Verbatim ==========================
%% ====================== Verbatim ==========================
\clearpage
\newpage
\subsection{Verbatim}

``The verbatim environment is a paragraph-making environment that gets
\LaTeX\ to print exactly what you type in. It turns \LaTeX\ into a typewriter
with carriage returns and blanks having the same effect that they would
on a typewriter.''\footnotemark

\begin{verbatim}
\begin{verbatim}
    text
\end{verbatim\}
\end{verbatim}


%% ========= Figure formatting with verbatim ===============
%% ========= Figure formatting with verbatim ===============
\noindent\begin{bf}Figure formatting with verbatim\end{bf}\\
%%
%% Making Figures in LaTeX
%% The [htb] part above advises LaTeX to put the figure "here"
%% or at the "top" or " bottom" of the page, with that order of preference.
%% http://www.sci.utah.edu/~macleod/latex/latex-figures.html
%%
\noindent The following figure leverages verbatim for proper formatting:
\vspace{-5mm}
\begin{center}
\begin{figure}[htb]
\begin{quote}
\small
\begin{Verbatim}[frame=none]
gis/raster/
  dnr/
    map_library/
    plats/
      SP/YYYYMMDD/*.pdf               # indexed
      HI/YYYYMMDD/*.pdf               # Indexed
      ASLS/YYYYMMDD/*.pdf             # Indexed
    recorded-plats/
      YYYYMMDD/*.pdf
  blm/
    easements_17b/YYYYMMDD/*.pdf      # indexed
    mtp/YYYYMMDD/*.pdf                # non-indexed
    usrs/YYYYMMDD/*.pdf               # indexed
    usrs-notes/YYYYMMDD/*.pdf         # indexed
    uss/YYYYMMDD/*.pdf                # indexed
    uss-notes/YYYYMMDD/*.pdf          # indexed
    usms/YYYYMMDD/*.pdf               # indexed
    usms-notes/YYYYMMDD/*.pdf         # indexed
  usgs/
    drg/
      collared/ 
        250K/
        63K/
        25K/
        24/
      decollared/
      tools/
      missing\_data/
    dem/
    doq/
    topo/
\end{Verbatim}
\normalsize
\end{quote}
\caption{File and Directory Structure}
\label{fig:dirlayout}
\end{figure}
\end{center}



%% ===================== Version Number ==========================
%% ===================== Version Number ==========================
\clearpage
\newpage
\subsection{Version Number}
It is often desirable to add a version number to a document for
tracking or revision control.  CVS or Subversion users can 
use the \cmd{rcs} or \cmd{svnInfo} packages for inline version
information.\\


\noindent\FIXME{Need to complete details here}



%% ======================== Endnotes =============================
%% ======================== Endnotes =============================
\clearpage
\newpage
\setcounter{secnumdepth}{0}
\section{Endnotes}

\begin{enumerate}

%% ====================== LaTeX Verbatim =========================
\item \LaTeX\ verbatim\\
\href{http://www.kfunigraz.ac.at/~binder/texhelp/ltx-79.html}
{http://www.kfunigraz.ac.at/~binder/texhelp/ltx-79.html}

\end{enumerate}


%% ======================== Appendix =============================
%% ======================== Appendix =============================
%%
%% This will add a standard non-numbered Appendix to the document.
%% The next section Appendix chnages secnumdepth such that the Appendix
%% is not numbered but still displayed in the table of contents (TOC).
%%
%% Adding unnumbered sections to TOC
%% http://tex.stackexchange.com/questions/11668/adding-unnumbered-sections-to-toc
%% 
%% \section*{Appendix}
%% Need content here.
%%
\setcounter{secnumdepth}{0}
\section{Appendix}


%% ============= LaTeX - Wikibooks ===============
A Guide to \LaTeX\\\
\href{http://www.astro.rug.nl/~kuijken/latex.html}
{http://www.astro.rug.nl/~kuijken/latex.html}
\\
\\
\LaTeX\ - From Wikibooks,the open-content textbooks collection\\
\href{http://en.wikibooks.org/wiki/LaTeX}{http://en.wikibooks.org/wiki/LaTeX}
\\
\\
\LaTeX\ Notes\\
\href{http://luke.breuer.com/time/item/LaTeX\_Notes/180.aspx}{http://luke.breuer.com/time/item/LaTeX\_Notes/180.aspx}

\end{document}

%% Local Variables:
%% fill-column: 78
%% mode: auto-fill
%% compile-command: "make"
%% End:
