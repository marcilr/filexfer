%% -*- Mode: LaTeX -*-
%%
%% introduction.tex
%% Created Fri Jul  1 08:53:27 AKDT 2016
%% by Raymond E. Marcil <rmarcil@gci.com>
%%
%% Introduction for file transfer jobs
%%



%% ====================== Introduction ===========================
%% ====================== Introduction ===========================
%% ====================== Introduction ===========================
\newpage
\setcounter{secnumdepth}{2}
\section{Introduction}
\fancyhead[LE,RO]{INTRODUCTION}
FileXfer is a custom application that GCI OSS built whose primary
purpose is to transfer files from point A to point B (most of the
time via point C/itself).  It can also be used to load the
collected data into a database if the data fits within the
constraints of MySQL Load Data Infile SQL Syntax.  It supports
FTP and SFTP for both gets and puts, and it also has limited
support for HTTP gets (this feature is used to collect weather
camera images off of the Terra mountain top sites for the
FAA).\footnote{FileXfer.txt:3, GCI Network Services, OSS Mark
Blum, Spring 2016}\\
\\
FileXfer can also be used to prune the source server's target
files to a certain number of days.  And while the default is
to keep the source file time, this feature can be toggle off
on a per job basis, resulting in the files having the transfer
time instead as some customers prefer to know when the file
was dropped off and not when the file was generated.  For
performance reasons there is a cutoff feature as well which
defaults to 5 days for new jobs.  FileXfer will not look more
than the cutoff days back to see if a file should be collected
and/or exported.  FileXfer jobs are also capable of running
in audit mode, in which FileXfer will log all of the transfers
but it won't physically transfer anything.  This feature can
be useful to get a feed caught up without transfering a bunch
of files around if for whatever reason the backlog of files
doesn't need to be processed by any customers.\\
\\
FileXfer logs all file transfers and any errors.  However, it
is not considered an error if there are no files to collect.
FileXfer also supports monitoring of transfer jobs, and can
generate an alert for any reason that you can articulate with
SQL.  Some examples include late and/or missing files, load
queue too large, file too small, no files transfered for a
certain interval, etc.  The monitoring supports internal
only alerts, TAC visibile alerts, and/or emailing the alerts.
The email feature also supports sending of texts to cell
phones.\\
\\
FileXfer will re-transfer a file if either the size and/or
source file time changes, as that signals something about the
file has changed.  Some feeds leverage this concept as they
may use a static filename in which the data is simply re-writen
to same exact file at regular intervals.\\
\\
The main FileXfer app server is
\texttt{prod-prov4-cdr1.operations.gci.com}
(\texttt{192.168.161.47}).  The ''ACS'' / Project Seward
FileXfer app server (which contains only ACS/Project
Serward related jobs) is the SPS2 OSS Test app server,
\texttt{osstest-em-provisioning.operations.gci.com}
(\texttt{192.168.56.4}; public IP \texttt{66.223.155.33}).


