%% -*- Mode: LaTeX -*-
%%
%% filexfer.tex
%% Created Mon Aug 22 10:20:54 AKDT 2016
%% by Raymond E. Marcil <rmarcil@gci.com>
%%
%% Filexfer
%%

%% ======================== FileXfer ============================
%% ======================== Filexfer ============================
%% ======================== Filexfer ============================
\subsection{FileXfer}
\fancyhead[LE,RO]{OPERATION}

There are times it is desirable to run FileXfer outside of the
FileXfer GUI app on Relevance.  Perhaps when it is desirable to
execute a single file transfer job. To do this login to 
GCI Network Services, OSS \texttt{prod-prov4-cdr1.operations.gci.com}
(\texttt{192.168.161.47}).\\
\\
The \texttt{filexfer} command is available from the command line.
It has relative straightforward syntax:

\begin{verbatim}
[root@prod-prov4-cdr1 usage]# /usr/bin/filexfer
Usage:
    filexfer.plx -c configfile -t {get|put} [options]

[root@prod-prov4-cdr1 usage]# cd ~
[root@prod-prov4-cdr1 ~]# /usr/bin/filexfer
Usage:
    filexfer.plx -c configfile -t {get|put} [options]

[root@prod-prov4-cdr1 ~]# /usr/bin/filexfer --help
Usage:
    filexfer.plx -c configfile -t {get|put} [options]

Arguments:
    -c, --configfile
        Specify the configuration file to load. Must be in YAML format.

    -t, --transfertype
        One of "get" or "put". Get jobs download files and put jobs upload
        files.

Options:
    -d, --piddir
        Directory where the pid file will be written. Defaults to
        /var/run/filexfer.

    --db
        Sets the database connection parameters. Valid keys are: server
        (default localhost), port (default 3306), driver (default mysql),
        uid, pwd, database, and table. Specify tags as key/value pairs,
        e.g.:

            --db server=localhost --db database=filexfer

    -e, --evengehost
        Address of the Evenge web server. Used to send indicators and events
        to the NMS system.

    --evengetimeout
        Timeout in seconds for communicating with the Evenge web server.
        Defaults to 10.

    -f, --cachefile
        Template cache file location. Defaults to
        /var/lib/filexfer/filexfer.kch.

    -h, --help
        Output this documentation.

    -m, --maxchildren
        Maximum number of child processes to spawn. Defaults to 50.

    -p, --pidfile
        PID file name. This will be appended with a ".pid" suffix.

    -r, --resource
        Resource name of this application. Used in indicator and event
        messages sent to the NMS system.

    --verbose, -v
        Log to the screen at increasingly verbose levels. This option may be
        repeated multiple times to increase the log level. For example, "-v"
        logs at info level, "-vv" logs at debug level, and "-vvv" logs at
        trace level.

[root@prod-prov4-cdr1 ~]# 
\end{verbatim}

\noindent The files transfered \texttt{filexfer} are under the
\texttt{/data/usage/} directory.

\begin{verbatim}
$ ssh hnmadm@prod-prov4-cdr1
hnmadm@192.168.161.47's password: 
Last login: Fri Aug 19 15:36:27 2016 from 10.103.193.217
[hnmadm@prod-prov4-cdr1 ~]$ cd /data/usage/
[hnmadm@prod-prov4-cdr1 usage]$ ls
AAA01       GA05  GA23  GA44  GA68  GC28  GC50    HAR01    P810     SPB02
ACS01       GA06  GA24  GA45  GA71  GC30  GC51    HCR01    P811     SPS02
AUB01       GA07  GA26  GA46  GA73  GC31  GC52    HMSC01   RBB01    SWPP01
BAL01       GA08  GA27  GA47  GA76  GC34  GC66    INC01    reports  SYNIVERSE
CAR01       GA09  GA28  GA48  GA77  GC35  GC87    INC02    RXNET01  TAL01
CBM01       GA10  GA29  GA49  GC05  GC36  GC88    INC03    SADC5E   TCS01
CDMA        GA11  GA30  GA50  GC15  GC37  GC89    IOP01    SBC01    TMP01
cdrReports  GA12  GA32  GA51  GC16  GC38  GC90    KUL01    SDC5E    TSUNAMI
CON01       GA13  GA33  GA52  GC17  GC39  GC91    legacy   SDE01    UNG01
CSKY01      GA14  GA34  GA53  GC18  GC41  GC93    LGCBX01  SDE01L   UNK01
CSKY02      GA15  GA35  GA55  GC19  GC42  GC94    MSW01    SDE01P   VORTEX01
DIM01       GA16  GA36  GA56  GC20  GC43  GC95    MTAS01   SDE02    WIFI01
EOS01       GA17  GA37  GA57  GC21  GC44  GC97    MUK01    SDE02L   WISP01
EPG01       GA18  GA38  GA58  GC22  GC45  GC98    Nov-18   SDE02P   WPS01
EPG02       GA19  GA39  GA59  GC23  GC46  GCT1    NRTRADE  SGSN01
GA01        GA20  GA41  GA65  GC24  GC47  GCT2    OCR01    SHU01
GA03        GA21  GA42  GA66  GC26  GC48  GGSN01  OSS01    SMSC01
GA04        GA22  GA43  GA67  GC27  GC49  GSM     P8       SPB01
[hnmadm@prod-prov4-cdr1 usage]$ sudo -s
Password:
[root@prod-prov4-cdr1 archive]#

\end{verbatim}

\noindent The directory names under \texttt{/data/usage/} map to the
\texttt{neName} column in the \texttt{filexfer.jobs} table that
is displayed in the FileXfer GUI application under the
File Transfer Jobs display in the \texttt{NE ID} field.\\
\\
The \texttt{filexfer get} jobs currently executing can be inspected with:

\begin{verbatim}
[root@prod-prov4-cdr1 usage]# ps ax | grep get
17585 ?        S      0:00 /usr/bin/filexfer \
-c /etc/filexfer/filexfer-get.conf -t get P8
17586 ?        S      0:00 /usr/bin/filexfer \
-x /etc/filexfer/filexfer-get.conf -t get P811
17588 ?        SN     0:00 /usr/bin/filexfer \
-c /etc/filexfer/filexfer-get.conf -t get SMSC01
17593 ?        SN     0:00 /usr/bin/filexfer \
-c /etc/filexfer/filexfer-get.conf -t get KUL01
17594 ?        SN     0:00 /usr/bin/filexfer \
-c /etc/filexfer/filexfer-get.conf -t get CAR01
24286 ?        S      0:00 /usr/bin/filexfer \
-c /etc/filexfer/filexfer-get.conf -t get GCT2
[root@prod-prov4-cdr1 usage]#
\end{verbatim}

\noindent The \texttt{filexfer put} jobs currently executing can be
inspected with:

\begin{verbatim}
[root@prod-prov4-cdr1 usage]# ps ax | grep put
...
 3863 ?        S      0:00 /usr/bin/filexfer \
-c /etc/filexfer/filexfer-put.conf -t put GC93
 3864 ?        S      0:00 /usr/bin/filexfer \
-c /etc/filexfer/filexfer-put.conf -t put GC94
 3866 ?        S      0:00 /usr/bin/filexfer \
-c /etc/filexfer/filexfer-put.conf -t put GC37
 3867 ?        S      0:00 /usr/bin/filexfer \
-c /etc/filexfer/filexfer-put.conf -t put GC95
 3869 ?        S      0:00 /usr/bin/filexfer \
-c /etc/filexfer/filexfer-put.conf -t put GA32
 3870 ?        S      0:00 /usr/bin/filexfer \
...
[root@prod-prov4-cdr1 usage]#
\end{verbatim}

\newpage
\noindent The same filexfer syntax displayed in the process list
can be using to execute an individual job to transfer files:

\begin{verbatim}
[root@prod-prov4-cdr1 ~]# /usr/bin/filexfer
Usage:
    filexfer.plx -c configfile -t {get|put} [options]

[root@prod-prov4-cdr1 ~]


[root@prod-prov4-cdr1 ~]# /usr/bin/filexfer -c /etc/filexfer/filexfer-put.conf -t put GC34
\end{verbatim}
