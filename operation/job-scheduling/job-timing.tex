%% -*- Mode: LaTeX -*-
%%
%% job-timing.tex
%% Created Thu Aug 11 08:15:46 AKDT 2016
%% by Raymond E. Marcil <rmarcil@gci.com>
%%
%% Job Timing for file transfer jobs
%%


%% =================== Job Timing ================================
%% =================== Job Timing ================================
\subsubsection{Job Timing}
The parent \texttt{filexfer} script is responsible for spawning
child processes for each job. Since a large number of jobs can be
scheduled at any given interval, the parent process limits how
many children can run concurrently.  As long as the limit is
reached and more jobs need to be spawned, the parent process must
stay alive. Since this may take longer than 1 minute, it is
possible for \texttt{filexfer} to miss certain scheduling
intervals.\\
\\
For example, if 500 jobs are scheduled to run at the top of every
hour \texttt{(0 * * * *)} and the maximum child process limit is
50, there is a good chance \texttt{filexfer} will not execute any
jobs scheduled to run at 1 minute past the hour
\texttt{(1 * * * *)}.  The best way to avoid this is to use
\texttt{0}, \texttt{15}, \texttt{30}, or \texttt{45} in the
minute field of the job schedule.  These intervals are always
executed.\footnote{\href{http://oss-wiki.operations.gci.com/dev/index.php/Scheduling\_Usage\_Collection\_Jobs}{Job
    Timing}}

